\documentclass{article}
\usepackage[utf8]{inputenc}

% Включение переносов для русского и английского языков
\usepackage[english,russian]{babel}

% Начинать первый параграф раздела следует с красной строки
\usepackage{indentfirst}

% Выбор внутренней TEX−кодировки
\usepackage [T2A]{fontenc}

\usepackage{amssymb}
\usepackage{amsmath}
\usepackage{cmap}

\usepackage{geometry} % Меняем поля страницы
\geometry{left=2cm}% левое поле
\geometry{right=1.5cm}% правое поле
\geometry{top=2cm}% верхнее поле
\geometry{bottom=2cm}% нижнее поле

\usepackage{listings}

\begin{document}
\subsection{Knuth-Morris-Pratt algo}
Pattern matching algorithm

\paragraph{Border}
\textit{Border}~---
    border of $P$ is
    $ |\alpha| $ where
    $\alpha \sqsupset P$ and $\alpha \sqsubset P$

\paragraph{Prefix function}
$$
    L(P, j) =
        \max{0 \leq k \leq j - 1}
        | P[1:k] \sqsupset P[1:j]
$$

Prefix function in code:
\begin{lstlisting}
std::vector<size_t>
ComputeStringPrefixFunction(
    const std::string& str
);
\end{lstlisting}

\paragraph{Knuth-Morris-Pratt algo}
C++ realisation:
\begin{lstlisting}
std::vector<size_t>
KnuthMorrisPrattFind(
    const std::string& pattern
    , const std::string& where
);
\end{lstlisting}

\end{document}
