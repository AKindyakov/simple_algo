\documentclass{article}
\usepackage[utf8]{inputenc}

% Включение переносов для русского и английского языков
\usepackage[english,russian]{babel}

% Начинать первый параграф раздела следует с красной строки
\usepackage{indentfirst}

% Выбор внутренней TEX−кодировки
\usepackage [T2A]{fontenc}

\usepackage{amssymb}
\usepackage{amsmath}
\usepackage{cmap}

\usepackage{geometry} % Меняем поля страницы
\geometry{left=2cm}% левое поле
\geometry{right=1.5cm}% правое поле
\geometry{top=2cm}% верхнее поле
\geometry{bottom=2cm}% нижнее поле

\usepackage{listings}

\begin{document}
\subsection{Longest common prefix (lcp)}
$$
    lcp(S_1, S_2, S_3, \ldots S_N) = longest(\alpha) | \alpha \sqsupset S_i, i = 1:N $$

\subsection{Z--Function}
$$
    Z(S) = array z[|S|], where z[i] = |lcp(S, S[i:])|
$$

\subsubsection{Native algo}
$$
    O(N^2)
$$

\subsubsection{Few optimization -- linear}
$$
    O(N)
$$
Для ускорения, можно достаточно просто получить на каждом шаге некоторую оценку
снизу для значения. Для этого помимо бегущего итератора $f$ по строке будем
поддерживать еще один итератор идущий следом $s$, так чтобы префикс второго
итератора $s$ ``накрывал'', тем самым можно получить позицию $f' = s - f$.
Значение функции в которой даст оценку для $f$. Если $z[f'] + f' \leq z[s]$, то
$z[f] = z[f']$ иначе $z[f] > z[f']$, точное значение получаем перебором. После
обновляем $s = f$.

После реализации кода выяснилось много нюансов, в конечном виде алгорит выглядит
так:
\paragraph{Переменные}
\begin{itemize}
    \item $T$ - исходная строка
    \item $z$ - массив результата, причем $size(z) == |T|$
    \item $f$ - впереди идущий индекс, на каждом шаге для него считается
значение функции.
    \item $s$ - позади идущий индекс, для него значение функции уже известно.
    \item $ff = f - s$
\end{itemize}

\paragraph{Вспомогательные функции}
\begin{itemize}
    \item $findFirstMissmatch(T, s, f)$ - итерируясь по идесксам в досль строки
находит первую позицию для второго индекса когда символы в строке различаются.
Например для аргументов ``abcdabce'', 0, 4 выход 7.
    В коде:
    \begin{lstlisting}
inline size_t
findFirstMissmatch(
    const std::string str,
    size_t s,
    size_t f
);
    \end{lstlisting}
\end{itemize}

\paragraph{Алгоритм}
\begin{enumerate}
    \item Пишем значение для элемента $z[0] = |T|$
    \item Пишем значение для элемента $z[1] = findFirstMissmatch(T, 0, 1);$
    \item $s = 1$, $f = 2$
    \item Если $s + z[s] <= f$, то мы ничего не выигрываем и ищем $z[f]$ простым
сканированием, $z[f] = findFirstMissmatch(T, 0, f)$, переходим к п.7.
    \item Если $f + z[ff] > s + z[s]$, то $z[f] = findFirstMissmatch(T, z[ff], s + z[s])$
, переходим к п.7.
    \item Если $f + z[ff] \leq s + z[s]$, то $z[f] = z[ff]$
    \item Далее проверяем что $z[s] < |T|$, нормируем
\end{enumerate}

В коде:
\begin{lstlisting}
std::vector<size_t>
zFunction(
    const std::string& str
);
\end{lstlisting}

\paragraph{Некоторые тонкости доказательства}
\begin{itemize}
    \item Из определения следует, что:
        $T[:z[s]] == T[s:(s + z[s])]$. Это важно помнить.
    \item Если $f$ целиком накрывается $z[s]$, т.е. $s + z[s] > f$, то справедливо:
    \begin{itemize}
        \item $T[f:(s + z[s])] == T[ff:z[s]]$
    \end{itemize}
\end{itemize}

\subsection{Pattern matching algo based on Z--Function}
$$
    rezult = Z(P + S)[i] \geq |P|, i = (|P| + 1) : |P + S|
$$
\textit{Sentinel symbol}~---

\end{document}
